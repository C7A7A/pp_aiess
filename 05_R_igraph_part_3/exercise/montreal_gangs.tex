% Options for packages loaded elsewhere
\PassOptionsToPackage{unicode}{hyperref}
\PassOptionsToPackage{hyphens}{url}
%
\documentclass[
]{article}
\usepackage{amsmath,amssymb}
\usepackage{iftex}
\ifPDFTeX
  \usepackage[T1]{fontenc}
  \usepackage[utf8]{inputenc}
  \usepackage{textcomp} % provide euro and other symbols
\else % if luatex or xetex
  \usepackage{unicode-math} % this also loads fontspec
  \defaultfontfeatures{Scale=MatchLowercase}
  \defaultfontfeatures[\rmfamily]{Ligatures=TeX,Scale=1}
\fi
\usepackage{lmodern}
\ifPDFTeX\else
  % xetex/luatex font selection
\fi
% Use upquote if available, for straight quotes in verbatim environments
\IfFileExists{upquote.sty}{\usepackage{upquote}}{}
\IfFileExists{microtype.sty}{% use microtype if available
  \usepackage[]{microtype}
  \UseMicrotypeSet[protrusion]{basicmath} % disable protrusion for tt fonts
}{}
\makeatletter
\@ifundefined{KOMAClassName}{% if non-KOMA class
  \IfFileExists{parskip.sty}{%
    \usepackage{parskip}
  }{% else
    \setlength{\parindent}{0pt}
    \setlength{\parskip}{6pt plus 2pt minus 1pt}}
}{% if KOMA class
  \KOMAoptions{parskip=half}}
\makeatother
\usepackage{xcolor}
\usepackage[margin=1in]{geometry}
\usepackage{color}
\usepackage{fancyvrb}
\newcommand{\VerbBar}{|}
\newcommand{\VERB}{\Verb[commandchars=\\\{\}]}
\DefineVerbatimEnvironment{Highlighting}{Verbatim}{commandchars=\\\{\}}
% Add ',fontsize=\small' for more characters per line
\usepackage{framed}
\definecolor{shadecolor}{RGB}{248,248,248}
\newenvironment{Shaded}{\begin{snugshade}}{\end{snugshade}}
\newcommand{\AlertTok}[1]{\textcolor[rgb]{0.94,0.16,0.16}{#1}}
\newcommand{\AnnotationTok}[1]{\textcolor[rgb]{0.56,0.35,0.01}{\textbf{\textit{#1}}}}
\newcommand{\AttributeTok}[1]{\textcolor[rgb]{0.13,0.29,0.53}{#1}}
\newcommand{\BaseNTok}[1]{\textcolor[rgb]{0.00,0.00,0.81}{#1}}
\newcommand{\BuiltInTok}[1]{#1}
\newcommand{\CharTok}[1]{\textcolor[rgb]{0.31,0.60,0.02}{#1}}
\newcommand{\CommentTok}[1]{\textcolor[rgb]{0.56,0.35,0.01}{\textit{#1}}}
\newcommand{\CommentVarTok}[1]{\textcolor[rgb]{0.56,0.35,0.01}{\textbf{\textit{#1}}}}
\newcommand{\ConstantTok}[1]{\textcolor[rgb]{0.56,0.35,0.01}{#1}}
\newcommand{\ControlFlowTok}[1]{\textcolor[rgb]{0.13,0.29,0.53}{\textbf{#1}}}
\newcommand{\DataTypeTok}[1]{\textcolor[rgb]{0.13,0.29,0.53}{#1}}
\newcommand{\DecValTok}[1]{\textcolor[rgb]{0.00,0.00,0.81}{#1}}
\newcommand{\DocumentationTok}[1]{\textcolor[rgb]{0.56,0.35,0.01}{\textbf{\textit{#1}}}}
\newcommand{\ErrorTok}[1]{\textcolor[rgb]{0.64,0.00,0.00}{\textbf{#1}}}
\newcommand{\ExtensionTok}[1]{#1}
\newcommand{\FloatTok}[1]{\textcolor[rgb]{0.00,0.00,0.81}{#1}}
\newcommand{\FunctionTok}[1]{\textcolor[rgb]{0.13,0.29,0.53}{\textbf{#1}}}
\newcommand{\ImportTok}[1]{#1}
\newcommand{\InformationTok}[1]{\textcolor[rgb]{0.56,0.35,0.01}{\textbf{\textit{#1}}}}
\newcommand{\KeywordTok}[1]{\textcolor[rgb]{0.13,0.29,0.53}{\textbf{#1}}}
\newcommand{\NormalTok}[1]{#1}
\newcommand{\OperatorTok}[1]{\textcolor[rgb]{0.81,0.36,0.00}{\textbf{#1}}}
\newcommand{\OtherTok}[1]{\textcolor[rgb]{0.56,0.35,0.01}{#1}}
\newcommand{\PreprocessorTok}[1]{\textcolor[rgb]{0.56,0.35,0.01}{\textit{#1}}}
\newcommand{\RegionMarkerTok}[1]{#1}
\newcommand{\SpecialCharTok}[1]{\textcolor[rgb]{0.81,0.36,0.00}{\textbf{#1}}}
\newcommand{\SpecialStringTok}[1]{\textcolor[rgb]{0.31,0.60,0.02}{#1}}
\newcommand{\StringTok}[1]{\textcolor[rgb]{0.31,0.60,0.02}{#1}}
\newcommand{\VariableTok}[1]{\textcolor[rgb]{0.00,0.00,0.00}{#1}}
\newcommand{\VerbatimStringTok}[1]{\textcolor[rgb]{0.31,0.60,0.02}{#1}}
\newcommand{\WarningTok}[1]{\textcolor[rgb]{0.56,0.35,0.01}{\textbf{\textit{#1}}}}
\usepackage{graphicx}
\makeatletter
\def\maxwidth{\ifdim\Gin@nat@width>\linewidth\linewidth\else\Gin@nat@width\fi}
\def\maxheight{\ifdim\Gin@nat@height>\textheight\textheight\else\Gin@nat@height\fi}
\makeatother
% Scale images if necessary, so that they will not overflow the page
% margins by default, and it is still possible to overwrite the defaults
% using explicit options in \includegraphics[width, height, ...]{}
\setkeys{Gin}{width=\maxwidth,height=\maxheight,keepaspectratio}
% Set default figure placement to htbp
\makeatletter
\def\fps@figure{htbp}
\makeatother
\setlength{\emergencystretch}{3em} % prevent overfull lines
\providecommand{\tightlist}{%
  \setlength{\itemsep}{0pt}\setlength{\parskip}{0pt}}
\setcounter{secnumdepth}{-\maxdimen} % remove section numbering
\ifLuaTeX
  \usepackage{selnolig}  % disable illegal ligatures
\fi
\IfFileExists{bookmark.sty}{\usepackage{bookmark}}{\usepackage{hyperref}}
\IfFileExists{xurl.sty}{\usepackage{xurl}}{} % add URL line breaks if available
\urlstyle{same}
\hypersetup{
  pdftitle={Montreal Gangs},
  hidelinks,
  pdfcreator={LaTeX via pandoc}}

\title{Montreal Gangs}
\author{}
\date{\vspace{-2.5em}2024-04-17}

\begin{document}
\maketitle

\hypertarget{data}{%
\subsection{1. Data}\label{data}}

Network representing relationships between gangs, obtained from Montreal
Police's central intelligence database, spanning 2004 to 2007. Nodes are
gangs, and an edge represents some kind of relationship between two
gangs (as elicited from interviews with gang members).

\begin{Shaded}
\begin{Highlighting}[]
\NormalTok{data }\OtherTok{\textless{}{-}} \FunctionTok{read.csv}\NormalTok{(}\StringTok{"montreal/MONTREALGANG.csv"}\NormalTok{, }\AttributeTok{header =} \ConstantTok{TRUE}\NormalTok{, }\AttributeTok{row.names =} \DecValTok{1}\NormalTok{)}

\NormalTok{adj\_matrix }\OtherTok{\textless{}{-}} \FunctionTok{as.matrix}\NormalTok{(data)}
\NormalTok{net }\OtherTok{\textless{}{-}} \FunctionTok{graph\_from\_adjacency\_matrix}\NormalTok{(adj\_matrix, }\AttributeTok{mode =} \StringTok{"max"}\NormalTok{, }\AttributeTok{weighted =} \ConstantTok{TRUE}\NormalTok{, }\AttributeTok{diag =} \ConstantTok{FALSE}\NormalTok{)}
\FunctionTok{plot}\NormalTok{(net, }\AttributeTok{layout =} \FunctionTok{layout\_nicely}\NormalTok{(net))}
\end{Highlighting}
\end{Shaded}

\includegraphics{montreal_gangs_files/figure-latex/unnamed-chunk-2-1.pdf}

\hypertarget{node-centrality}{%
\subsection{2. Node centrality}\label{node-centrality}}

\begin{Shaded}
\begin{Highlighting}[]
\NormalTok{degree\_centrality }\OtherTok{\textless{}{-}} \FunctionTok{degree}\NormalTok{(net)}
\NormalTok{nodes\_size }\OtherTok{\textless{}{-}} \DecValTok{10} \SpecialCharTok{+} \DecValTok{10} \SpecialCharTok{*}\NormalTok{ degree\_centrality }\SpecialCharTok{/} \FunctionTok{max}\NormalTok{(degree\_centrality)}
\FunctionTok{plot}\NormalTok{(net, }\AttributeTok{layout =} \FunctionTok{layout\_nicely}\NormalTok{(net), }\AttributeTok{vertex.size=}\NormalTok{nodes\_size)}
\end{Highlighting}
\end{Shaded}

\includegraphics{montreal_gangs_files/figure-latex/unnamed-chunk-3-1.pdf}

\hypertarget{degree-distribution}{%
\subsection{3. Degree distribution}\label{degree-distribution}}

\begin{Shaded}
\begin{Highlighting}[]
\NormalTok{degree\_distr }\OtherTok{=} \FunctionTok{degree\_distribution}\NormalTok{(net)}
\FunctionTok{hist}\NormalTok{(}
\NormalTok{  degree\_centrality, }
  \AttributeTok{breaks =} \DecValTok{20}\NormalTok{, }
  \AttributeTok{col =} \StringTok{"skyblue"}\NormalTok{, }
  \AttributeTok{main =} \StringTok{"Degree Distribution"}\NormalTok{, }
  \AttributeTok{xlab =} \StringTok{"Degree"}\NormalTok{, }
  \AttributeTok{ylab =} \StringTok{"Frequency"}
\NormalTok{)}
\end{Highlighting}
\end{Shaded}

\includegraphics{montreal_gangs_files/figure-latex/unnamed-chunk-4-1.pdf}

\hypertarget{longest-path-in-net}{%
\subsection{4. Longest path in net}\label{longest-path-in-net}}

\begin{Shaded}
\begin{Highlighting}[]
\NormalTok{diamenter\_path }\OtherTok{=} \FunctionTok{get\_diameter}\NormalTok{(net)}

\NormalTok{ecol }\OtherTok{\textless{}{-}} \FunctionTok{rep}\NormalTok{(}\StringTok{"gray80"}\NormalTok{, }\FunctionTok{ecount}\NormalTok{(net))}
\NormalTok{ecol[}\FunctionTok{E}\NormalTok{(net, }\AttributeTok{path=}\NormalTok{diamenter\_path)] }\OtherTok{\textless{}{-}} \StringTok{"orange"} 

\NormalTok{ew }\OtherTok{\textless{}{-}} \FunctionTok{rep}\NormalTok{(}\DecValTok{2}\NormalTok{, }\FunctionTok{ecount}\NormalTok{(net))}
\NormalTok{ew[diamenter\_path] }\OtherTok{\textless{}{-}} \DecValTok{4}

\NormalTok{vcol }\OtherTok{\textless{}{-}} \FunctionTok{rep}\NormalTok{(}\StringTok{"gray40"}\NormalTok{, }\FunctionTok{vcount}\NormalTok{(net))}
\NormalTok{vcol[diamenter\_path] }\OtherTok{\textless{}{-}} \StringTok{"gold"}

\FunctionTok{plot}\NormalTok{(net, }
   \AttributeTok{vertex.color=}\NormalTok{vcol, }
   \AttributeTok{edge.color=}\NormalTok{ecol, }
   \AttributeTok{edge.width=}\NormalTok{ew, }
   \AttributeTok{edge.arrow.mode=}\DecValTok{0}\NormalTok{,}
   \AttributeTok{layout =}\NormalTok{ layout.fruchterman.reingold}
\NormalTok{)}
\end{Highlighting}
\end{Shaded}

\includegraphics{montreal_gangs_files/figure-latex/unnamed-chunk-5-1.pdf}

\hypertarget{communities}{%
\subsection{5. Communities}\label{communities}}

\begin{Shaded}
\begin{Highlighting}[]
\NormalTok{clp }\OtherTok{\textless{}{-}} \FunctionTok{cluster\_label\_prop}\NormalTok{(net)}
\FunctionTok{plot}\NormalTok{(clp, net)}
\end{Highlighting}
\end{Shaded}

\includegraphics{montreal_gangs_files/figure-latex/unnamed-chunk-6-1.pdf}

\end{document}
